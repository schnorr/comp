\documentclass{compiladores}
\begin{document}

\begin{center}
{\LARGE Lista de Exercícios \#10}\\
{\bf Geração de Código}
\end{center}

\bigskip

\begin{listanumerada}
\item \label{x} Transforme a gramática abaixo em um esquema de tradução que gere TAC.

  \begin{tabular}{llll}
    S  &  $\rightarrow$  &  x = E;            \\
    E  &  $\rightarrow$  &  E$_1$ + T         \\
    E  &  $\rightarrow$  &  E$_1$ - T         \\
    E  &  $\rightarrow$  &  T                 \\
    T  &  $\rightarrow$  &  T$_1$ * F         \\
    T  &  $\rightarrow$  &  T$_1$ / F         \\
    T  &  $\rightarrow$  &  F                 \\
    F  &  $\rightarrow$  &  ( E )             \\
    F  &  $\rightarrow$  &  \textbf{digit}    \\
    F  &  $\rightarrow$  &  \textbf{var}    \\
  \end{tabular}

\item Suponha que os operadores do código TAC ($+ - * /$) só suportem
  operandos do mesmo tipo e que existe somente dois tipos na
  linguagem: \texttt{int} e \texttt{float}. Altere o esquema de
  tradução do exercício~\ref{x} para utilizar apropriadamente os
  operadores TAC. Mostre exemplos que estressam a funcionalidade.

\item \label{y} Suponha que uma linguagem tenha apenas escopo
  global. Declarações de variáveis dentro de funções não são
  autorizadas nesta linguagem. Crie um esquema de tradução para
  calcular os endereços dessas variáveis em relação ao endereço base
  do segmento de dados. Esta informação deve ser mantida na tabela de
  símbolos.

\item \label{y1} Altere o esquema de tradução do exercício~\ref{y}
  para gerar TAC considerando diferentes entradas válidas de
  expressões aritméticas. O acesso ao conteúdo das variáveis é feito
  levando-se em conta o endereço base do segmento de dados.

\item \label{y2} Suponha que os operandos $+ - * /$ do código TAC não aceitam
  endereços como parâmetros. Por exemplo, o seguinte código TAC é
  ilegal. Na linha 1, temos um endereço no primeiro parâmetro do
  operando $*$; na linha 2, temos um endereço no segundo parâmetro do
  operando $+$:
  \begin{lstlisting}
    t0 = (fp+0) * 5.2
    t1 = t0 + (fp+8)
  \end{lstlisting}
  Sabendo desta limitação, altere o esquema de tradução resultante do
  exercício~\ref{y1} para que os valores em memória (parâmetros de
  endereçamento) sejam primeiramente colocados em registradores para
  que os operandos possam trabalhar sobre eles. Utilize o operando
  \texttt{load} do nosso TAC hipotético para carregar os valores da
  memória para os registradores. Exemplo:
  \begin{lstlisting}
    t0 = load (fp+0)
    t1 = load (fp+8)
    t2 = t0 * 5.2
    t3 = t2 + t1
  \end{lstlisting}

\item \label{a} Crie um esquema de tradução para a declaração de variáveis com
  escopo aninhado. Mostre o funcionamento com entradas
  válidas.

\item Incremente o esquema de tradução resultante do exercício~\ref{a}
  para a geração de código TAC para expressões aritméticas. Considere
  as limitações TAC apresentadas no exercício~\ref{y2}.

\item Como você implementaria a geração de código para o comando de
  atribuição em uma linguagem com escopo aninhado?

\item Considere uma linguagem orientada a objetos com o conceito de
  herança. Como você implementaria o endereçamento dos atributos de
  instância?
\end{listanumerada}
\end{document}

\item \label{x1} Transforme a gramática do exercício~\ref{x} em um esquema de
  tradução que cria uma árvore de derivação.

\item Repita o exercício~\ref{x1}, criando uma árvore sintática
  abstrata para qualquer entrada.

\item \label{y} Repita o exercício~\ref{x1}, criando um grafo acíclico
  direcionado.

\item \label{t} O esquema de tradução abaixo gera uma árvore sintática
  abstrata. Mostre seu funcionamento para uma série de entradas
  válidas.

\begin{tabular}{lll}
 E  &  $\rightarrow$  &  T \texttt{ \{ R.h = T.ptr; \} } R \texttt{ \{ E.ptr = R.s; \} }                                \\
 R  &  $\rightarrow$  &  + T \texttt{ \{ $R_1$.h = geraNo('+', R.h, T.ptr); \} } $R_1$ \texttt{ \{ R.s = $R_1$.s; \} }  \\
 R  &  $\rightarrow$  &  - T \texttt{ \{ $R_1$.h = geraNo('-', R.h, T.ptr); \} } $R_1$ \texttt{ \{ R.s = $R_1$.s; \} }  \\
 R  &  $\rightarrow$  &  $\epsilon$ \texttt{ \{ R.s = R.h; \} }                                                         \\
 T  &  $\rightarrow$  &  ( E ) \texttt{ \{ T.ptr = E.ptr; \} }                                                          \\
 T  &  $\rightarrow$  &  id \texttt{ \{ T.ptr = geraFolha(id, id.nome); \} }                                            \\
 T  &  $\rightarrow$  &  enum \texttt{ \{ T.ptr = geraFolha(num, num.val); \} }                                         \\
\end{tabular}

\item O que é um bloco básico? Por que alguns compiladores adotam um
  conceito de bloco básico diferente do tradicional? Discorra sobre as
  vantagens de cada abordagem e seu impacto nas etapas de otimização,
  por exemplo.

\item Construa o grafo de fluxo de controle para o código abaixo:
  
  \begin{lstlisting}
       stmt0
       while (i < 100) { stmt1 }
       stmt2
       if (x = y) { stmt3 } else { stmt4 }
       stmt5
  \end{lstlisting}

\item Construa o grafo de dependência de dados para o código abaixo:

  \begin{lstlisting}
       x = 0
       i = 1
       while (i < 100)
           if (a[i] > 0)
              then x = x + a[i]
           i = i + 1
       print x
    \end{lstlisting}

\item Qual a vantagem e a desvantagem da IR linear de código de um endereço?

\item \label{z} Transforme a gramática do exercício~\ref{x} em um esquema de
  tradução que gere código de um endereço.

\item Traduza uma série de entradas válidas utilizando o esquema de
  tradução do exercício~\ref{z}.

\item \label{y1} Altere o esquema de tradução criado no exercício~\ref{y},
  gerando código de um endereço. Utilize o grafo acíclico direcionado
  para aproveitar valores eventualmente já calculados.

\item Refaça o exercício~\ref{y1}, gerando código de três endereços.

\item O que é TAC?

\item Crie um esquema de tradução para gerar uma IR pós-fixada a
  partir de uma IR pré-fixada de expressões aritméticas. Apresente uma
  série de entradas válidas mostrando o seu funcionamento com uma
  analisador ascendente e outro analisador descendente. Explique as
  diferenças, caso existam, nos esquemas de tradução para cada um dos
  analisadores.

\item Existem três abordagens para implementar TAC em memória, ou
  seja, manter o código TAC em memória. Qual delas você acha mais
  apropriada para o projeto de compiladores?
\end{listanumerada}
\end{document}
