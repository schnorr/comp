\documentclass{compiladores}
\begin{document}
\begin{center}
{\LARGE Lista de Exercícios \#04}\\
{\bf Análise Descendente, Gramáticas LL(1) e Tabelas Preditivas}
\end{center}

\begin{listanumerada}
\item
Como sabemos quando uma gramática não é LL(1)?

\item A gramática seguinte não pode ser utilizada em uma análise
  preditiva descendente. Identifique e corrija o problema,
  reescrevendo a gramática. Mostre que a nova gramática, corrigida,
  satisfaz a condição LL(1). \\
  \begin{tabular}{rcl}
    L & $\rightarrow$ & Ra | Qba \\
    R & $\rightarrow$ & aba | caba | Rbc \\
    Q & $\rightarrow$ & bbc | bc \\
  \end{tabular}

\item A gramática seguinte satisfaz a condição LL(1)? Justifique a sua
  resposta. Se ela não satisfazer, reescreva-a como uma gramática
  LL(1) equivalente, capaz de reconhecer a mesma linguagem. \\
  \begin{tabular}{rcl}
    A & $\rightarrow$ & Ba \\
    B & $\rightarrow$ & dab | Cb \\
    C & $\rightarrow$ & cB | Ac \\
  \end{tabular}

\item Para cada uma das gramáticas seguintes, defina analisadores
  preditivos descendentes através da construção de tabelas preditivas
  de análise. Algumas dessas gramáticas precisam ser, inicialmente,
  fatoradas à esquerda ou ter sua recursão à esquerda eliminada.
  \begin{lista}
  \item $S \rightarrow 0\ S\ 1\ |\ 0\ 1$
  \item $S \rightarrow +\ S\ S\ |\ *\ S\ S\ |\ a$
  \item $S \rightarrow S\ (\ S\ )\ S\ |\ {\epsilon}$
  \item $S \rightarrow S + S\ |\ SS\ |\ (\ S\ )\ |\ S\ *\ |\ a$
  \item $S \rightarrow (\ L\ )\ |\ a$ e $L \rightarrow L, S\ |\ S$
  \item $S \rightarrow aSbS | bSaS | {\epsilon}$
  \item Gramática para expressões boleanas: \\
    \begin{tabular}{rcl}
      bexpr & $\rightarrow$ & bexpr {\bf or} bterm | bterm \\
      bterm & $\rightarrow$ & bfactor | bfactor \\
      bfactor & $\rightarrow$ & {\bf not} bfactor | {\bf (} bexpr {\bf )} | {\bf true} | {\bf false}
    \end{tabular}
  \end{lista}

\item É possível realizar a construção de um analisador preditivo
  descendente para a gramática $S \rightarrow SS+ |\ SS* |\ a$ utilizada
  para descrever as expressões pós-fixadas com o operador $a$? Quais
  são as modificações necessárias na gramática para efetuar tal tarefa?

\item \label{x1} A gramática seguinte satisfaz a condição LL(1)? Justifique a sua
  resposta. Qual tipo de linguagem ela descreve? \\
    \begin{tabular}{rcl}
      S & $\rightarrow$ & $(L) | p | q$ \\
      L & $\rightarrow$ & $L and S | L or S | S$ \\
    \end{tabular}

\item Implemente, em pseudo-código, um analisador recursivo
  descendente para a linguagem descrita no exercício~\ref{x1},
  transformando a gramática em LL(1) se necessário.

\item Seja L uma linguagem cujas sentenças são formadas de qualquer
  texto. A seguinte sentença faz parte desta linguagem: ``um (grande)
  animal chamado gato (podendo logicamente ser um leão ou um tigre
  \{que são cada vez mais raros\} ou ainda um tigre dente-de-sabre
  \{que está extinto [ ler mais em Kurten ], o que é uma vergonha\} ou
  leopardo) é um atraente (mas perigoso) amigo''. Você pode supor, por
  simplicidade, que o texto consiste apenas de letras e
  espaços. Defina uma gramática LL(1) que descreva esta linguagem e,
  por fim, construa uma tabela preditiva de análise descendente para a
  mesma. Mostre que a tabela funciona para uma entrada válida e uma
  entrada não válida.

\item Considere a seguinte gramática \\
    \begin{tabular}{rcl}
      expressão & $\rightarrow$ & átomo | lista \\
      átomo & $\rightarrow$ & número | idenficador \\
      lista & $\rightarrow$ & (sequência\_de\_expressões) \\
      sequência\_de\_expressões & $\rightarrow$ & expressão, sequência\_de\_expressão | expressão \\
    \end{tabular}
    \begin{lista}
      \item Fatore à esquerda esta gramática, se necessário, e elimine a recursão à esquerda, se existir.
      \item Construa uma tabela preditiva de análise descendente LL(1) para esta gramática.
      \item Mostre as ações do analisador LL(1) correspondente, considerando a entrada {\bf (a,(b,(2)),(c))}
    \end{lista}

\item Considerando a gramática \texttt{g1}. O conjunto dos símbolos
  terminais é \{a, b, c\}, o conjunto dos símbolos não-terminais é
  \{S, A, B, C\}, o símbolo inicial é S, e as produções são as
  seguintes: \\
  \begin{tabular}{rcl}
    S & $\rightarrow$ & cA | b \\
    A & $\rightarrow$ & cBC | bSA | a \\
    B & $\rightarrow$ & cc | Cb \\
    C & $\rightarrow$ & aS | ba \\
  \end{tabular}

  Considerando a gramática \texttt{g2}. O conjunto dos símbolos
  terminais é \{a, b\}, o conjunto dos símbolos não-terminais é \{S,
  A\}, o símbolo inicial é S, e as produções são as seguintes: \\
  \begin{tabular}{rcl}
    S & $\rightarrow$ & abA | aa \\
    A & $\rightarrow$ & bb | bS \\
  \end{tabular} \\
  Considerando a gramática \texttt{g3}. O conjunto dos símbolos
  terminais é \{a, b, c\}, o conjunto dos símbolos não-terminais é
  \{S, A, B\}, o símbolo inicial é S, e as produções são as seguintes:
  \\
  \begin{tabular}{rcl}
    S & $\rightarrow$ & AaS | B \\
    A & $\rightarrow$ & cS | $\epsilon$ \\
    B & $\rightarrow$ & b \\
  \end{tabular} \\

  \begin{lista}
    \item Para cada gramática \texttt{g1}, \texttt{g2} e \texttt{g3},
      fatore à esquerda se necessário, e elimine a recursão à esquerda
      se houver.
    \item Construa tabelas preditivas LL(1) para a análise descendente
      para cada gramática \texttt{g1}, \texttt{g2} e \texttt{g3}.
    \item Para a gramática \texttt{g1}, mostre os passos da análise de
      {\bf ccccba} utilizando a tabela construída.
    \item Para a gramática \texttt{g2}, mostre os passos da análise de
      {\bf abbb} utilizando a tabela construída.
    \item Para a gramática \texttt{g3}, mostre os passos da análise de
      {\bf acbab} utilizando a tabela construída.
  \end{lista}

\item Construa a tabela preditiva da seguinte gramática (já fatorada à
  esquerda e sem recursão à esquerda): \\
     \begin{tabularx}{\linewidth}{lcllcllcl}
     E  & $\rightarrow$ & TE'             \\
     E' & $\rightarrow$ & ATE' | $\epsilon$ \\
     T  & $\rightarrow$ & FT' \\
     T' & $\rightarrow$ & BFT' | $\epsilon$ \\
     F  & $\rightarrow$ & (E) | id \\
     A  & $\rightarrow$ & + | - \\
     B  & $\rightarrow$ & $*$ | $/$
     \end{tabularx}
     \begin{lista}
       \item Mostre os passos de análise para a entrada {\bf id / id - id}
       \item Caso exista ambiguidade, resolva o problema modificando a
         tabela. Justifique cada escolha.
       \item Mostre os passos de análise para {\bf id * id / id} e
         {\bf id + id - id}, mostrando o efeito das suas decisões.
     \end{lista}

\item Construa a tabela preditiva da seguinte gramática e mostre os passos de {\bf ibtibtaeaea} \\
    \begin{tabular}{lcl}
     S & $\rightarrow$ & iEtSS' | a \\
     S' & $\rightarrow$ & eS | $\epsilon$ \\
     E & $\rightarrow$ & b\\
     \end{tabular}

\item Contrua a tabela preditiva da seguinte gramática \\
  \label{y}
    \begin{tabular}{lcl}
     S & $\rightarrow$ & {\bf if (} E {\bf )} SS' | {\bf \{} S {\bf\}} | {\bf a} \\
     S' & $\rightarrow$ & {\bf else} S | $\epsilon$ \\
     E & $\rightarrow$ & b\\
     \end{tabular}
    \begin{lista}
      \item Mostre os passos de análise para {\bf if(b)\{if(b)a\}elsea }
      \item Mostre os passos de análise para {\bf if(b)\{if(b)\{if(b)a\}elsea\}else\{if(b)aelse\{a\}\}}
    \end{lista}
\item Explique o porquê da gramática seguinte não ser ambígua, fazendo
  referência ao exercício~\ref{y}. Construa a tabela
  preditiva desta gramática e mostre os passos de análise para {\bf  \{if\{a\}else\{a\}\}} \\
    \begin{tabular}{lcl}
     S & $\rightarrow$ & {\bf \{} A {\bf \}} \\
     A & $\rightarrow$ & {\bf if} SA' | {\bf a} \\
     A' & $\rightarrow$ & {\bf else} S \\
     \end{tabular}

\end{listanumerada}
\end{document}
