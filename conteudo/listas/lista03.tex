\documentclass{compiladores}
\begin{document}

\begin{center}
{\LARGE Lista de Exercícios \#03}\\
{\bf Transformações e Conjuntos Primeiro (First) e Sequência (Follow)}
\end{center}

\begin{listanumerada}
\item
\label{fatoracao-recursao}
Para cada uma das gramáticas a seguir, fatore à esquerda e/ou elimine
a recursão à esquerda, preparando-as para a construções de
analisadores recursivos preditivos e depois realize os passos de
derivações necessários para o reconhecimento das cadeias especificadas
em cada item (com exceção do último item, onde isto não é necessário).
\begin{lista}
\item $S \rightarrow 0\ S\ 1\ |\ 0\ 1$ com as cadeias $000111$ e $01011001$. 
\item $S \rightarrow +\ S\ S\ |\ *\ S\ S\ |\ a$ com as cadeiras $+*aaa$ e $+aa*aa$.
\item $S \rightarrow S\ (\ S\ )\ S\ |\ {\epsilon}$ com as cadeias $(()())$ e $((()())()())$.
\item $S \rightarrow S + S\ |\ SS\ |\ (\ S\ )\ |\ S\ *\ |\ a$ com as cadeias $(a+a)*a$ e $(a*a)a$.
\item $S \rightarrow (\ L\ )\ |\ a$ e $L \rightarrow L, S\ |\ S$ com a cadeia $((a,a),a,(a))$.
\item $S \rightarrow aSbS | bSaS | {\epsilon}$ com a cadeia $aabbab$.
\item A gramática para expressões boleanas: \\
  \begin{tabular}{rcl}
    bexpr & $\rightarrow$ & bexpr {\bf or} bterm | bterm \\
    bterm & $\rightarrow$ & bterm {\bf and} bfactor | bfactor \\
    bfactor & $\rightarrow$ & {\bf not} bfactor | {\bf (} bexpr {\bf )} | {\bf true} | {\bf false}
  \end{tabular}
\end{lista}

\item
Calcule os conjuntos Primeiro e Sequência para todas as gramáticas resultantes do exercício~\ref{fatoracao-recursao}.

\item
\label{expressoesposfixadas}
Considerando a gramática livre de contexto definida pelo conjunto de
produções $P = \{ S \rightarrow S\ S\ +\ |\ S\ S\ *\ |\ a\ \}$ que
descreve as expressões pós-fixadas com o operando \texttt{a}. É
possível, modificando esta gramática de alguma maneira, construir um
analisador preditivo?

\item
Calcule os conjuntos Primeiro e Sequência para a gramática apresentada no exercício~\ref{expressoesposfixadas}.

\item
Considerando a gramática definida pelo conjunto de produções $P = \{ E
\rightarrow E + E\ |\ E * E\ |\ -E\ |\ +E\ |\ ( E )\ |\ id\ \}$:
  \begin{lista}
  \item Retire a ambiguidade desta gramática, dando preferência para a
    operação de multiplicação.
  \item Remove a recursão à esquerda, se houver.
  \item Fatore totalmente a gramática, se necessário.
  \item Calcule os conjuntos Primeiro e Sequência da gramática modificada
    pelos passos anteriores.
  \end{lista}

\item
Calcule os conjuntos Primeiro and Sequência para cada uma das gramáticas seguintes:
\begin{lista}
\item $S \rightarrow AB$, $A \rightarrow c|{\epsilon}$, $B \rightarrow cbB|ca$
\item $S \rightarrow ABc$, $A \rightarrow a | \epsilon$, $B \rightarrow b | \epsilon$
\item $S \rightarrow Ab$, $A \rightarrow a | B | \epsilon$, $B \rightarrow b | \epsilon$
\item $S \rightarrow ABBA$, $A \rightarrow a | \epsilon$, $B \rightarrow b | \epsilon$
\item $S \rightarrow aSe | B$, $B \rightarrow bBe | C$, $C \rightarrow cBe | d$
\end{lista}

\item
Defina os conjuntos Primeiro e Sequência considerando a gramática seguinte: \\
  \begin{tabular}{rcl}
    E & $\rightarrow$ & {\bf -}E | {\bf (} E {\bf )} | V L \\
    L & $\rightarrow$ & {\bf -}E | $\epsilon$ \\
    V & $\rightarrow$ & {\bf id} S \\
    S & $\rightarrow$ & {\bf (} E {\bf )} | $\epsilon$
  \end{tabular}

\end{listanumerada}
\end{document}

